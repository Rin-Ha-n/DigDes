\section{Programmazione VHDL}
% =========================================== ENTITY =========================================
    \subsection{entity}
        \begin{lstlisting}[language=VHDL, escapeinside={(*@}{@*)}]
            entity <entity name> is
                port (
                    {<port_name> : <mode> <type>;}
                );
            end entity <entity name>;
        \end{lstlisting}

% =========================================== ARCHITECTURE ======================================
    \subsection{architecture}
        \begin{lstlisting}[language=VHDL, escapeinside={(*@}{@*)}, numberstyle=\tiny\color{gray}\highlightlines{3}{6}{yellow}]
            architecture <architecture_type> of <entity_name> is
                [type_declaration]
                    [subtype_declaration]
                    [constant_declaration]
                    [signal_declaration]
                    [component_declaration]
            begin
                -- codice di architettura
            end <architecture_type>;
        \end{lstlisting}


% ========================================= COMPONENTI =========================================
    \subsection{component}
        I componenti sono utilizzati per definire le porte di un'entità, in modo da poterla utilizzare in altre entità.
        \begin{lstlisting}[language=VHDL]
            component <component_name> is
                port (
                    {<port_name> : <mode> <type>;}
                );
            end component <component_name>;
        \end{lstlisting}


% ========================================= PROCESSI =========================================
    \subsection{Processi}
        I processi sono sezioni di codice che vengono eseguite ogni volta che un \textcolor{red}{Segnale sensibile} nella lista sensibile cambia di stato.
        % Inizia un codice blocco, con colori VHDL (vedi definizione in DigDes.tex), escape inside is used to use external commandstf
        \begin{lstlisting}[language=VHDL, escapeinside={(*@}{@*)}, numberstyle=\tiny\color{gray}\highlightlines{3}{7}{yellow}]
        process ((*@\textcolor{red}{clk, reset}@*))
            begin
                if reset = '1' then
                    -- codice di reset
                elsif rising_edge(clk) then
                    -- codice di clock
                end if;
            end process;
        \end{lstlisting}


% ============================================ TIPI =========================================
    \subsection{Tipi}
        \begin{itemize}
            \item \texttt{<architecture\_type>} = \texttt{Behavioral} \textbar{} \texttt{Structural} \textbar{} \texttt{RTL} \textbar{} \texttt{Dataflow} \textbar{} \texttt{Tb}
            \item \texttt{<mode>} = \texttt{in} \textbar{} \texttt{out} \textbar{} \texttt{inout}
            \item \texttt{<type>} = \texttt{bit} \textbar{} \texttt{bit\_vector} \textbar{} \texttt{std\_}\textcolor{red}{u}\texttt{logic} \textbar{} \texttt{std\_}\textcolor{red}{u}\texttt{logic\_vector} \textbar{} \texttt{integer} \textbar{} \texttt{boolean}
        \end{itemize}
        


% ----------------------------------------ARCHITECTURE TYPES ------------------------------------
        \subsubsection{\texttt{<architecture\_type>}}
            \textbf{Behavioral}: si occupa di descrivere il comportamento del circuito, senza preoccuparsi della struttura fisica.
            \begin{lstlisting}[language=VHDL]
                if rising_edge(clk) then
                    if A = '1' then
                    Y <= B;
                    end if;
                end if;
            \end{lstlisting}

            \textbf{Structural}: si occupa di descrivere la struttura fisica del circuito, utilizzando componenti e connessioni tra di essi.
            \begin{lstlisting}[language=VHDL]
                U1: and_gate port map (A => A, B => B, Y => Y1);
                U2: or_gate  port map (A =1, B => C, Y => Y);
            \end{lstlisting}

            \textbf{RTL}: si occupa di descrivere il circuito a livello di registro e logica combinatoria, utilizzando registri e porte logiche.
            \begin{lstlisting}[language=VHDL]
                if rising_edge(clk) then
                    reg1 <= A and B;
                    reg2 <= reg1 xor C;
                end if;
            \end{lstlisting}

            \textbf{Dataflow}: si occupa di descrivere il circuito a livello di flusso di dati, utilizzando porte logiche e segnali.
            \begin{lstlisting}[language=VHDL]
                Y <= (A and B) or (not C);
            \end{lstlisting}

            \textbf{Tb}: si occupa di descrivere il circuito a livello di testbench, utilizzando segnali di test e componenti di test.
            \begin{lstlisting}[language=VHDL]
                A <= '0'; wait for 10 ns;
                A <= '1'; wait for 10 ns;
                assert (Y = expected_value) report "Test failed" severity error;
            \end{lstlisting}

% ----------------------------------------MODE TYPES ------------------------------------
        \subsubsection{\texttt{<type>}}
               \begin{itemize}
                \item \texttt{bit}: rappresenta un singolo bit, con valori \texttt{'0'} e \texttt{'1'}.
                \item \texttt{bit\_vector}: rappresenta un vettore di bit, con valori \texttt{'0'} e \texttt{'1'}.
                \item \texttt{std\_logic}: rappresenta un singolo bit con valori \texttt{'0'}, \texttt{'1'}.
                \item \texttt{std\_logic\_vector}: rappresenta un vettore di std\_logic, con valori \texttt{'0'}, \texttt{'1'}.
                \item \texttt{std\_ulogic}: rappresenta un singolo bit con valori \texttt{'0'}, \texttt{'1'}, \texttt{'Z'} (alta impedenza) e \texttt{'X'} (indeterminato).
                \item \texttt{std\_ulogic\_vector}: rappresenta un vettore di std\_ulogic, con valori \texttt{'0'}, \texttt{'1'}, \texttt{'Z'} e \texttt{'X'}.
                \item \texttt{integer}: rappresenta un numero intero, con valori compresi tra $-2^{31}$ e $2^{31}-1$.
                \item \texttt{boolean}: rappresenta un valore booleano, con valori \texttt{true} e \texttt{false}.
               \end{itemize} 