\section{State machine}
    \begin{minipage}[t]{0.48\columnwidth}
        \vspace{0pt} % <-- ensures top alignment
        Le \textbf{Finite State Machine} (FSM) sono circuiti sequenziali che possono essere in uno stato tra un insieme finito di stati. 
        La transizione tra gli stati avviene in base a segnali di ingresso ed é regolata da segnali di clock e reset.
    \end{minipage}
    \hfill
    \begin{minipage}[t]{0.48\columnwidth}
        \vspace{0pt} % <-- ensures top alignment
        \includegraphics[width=\linewidth]{Images/FSMGeneralizzata.png}
    \end{minipage}%
    

% ==================================== FAMIGLIE FSM ====================================
    %---------- MEDWEJDJEW ----------
    \subsection{Medwedjew}
    \begin{minipage}[t]{0.48\columnwidth}
        \vspace{0pt} % <-- ensures top alignment
        Composta da una un blocco di logica combinatoria(G) che risolve la funzione desiderata e da un blocco di memoria(Z)
        che memorizza gli stati.\\ 
        => Gli Input e lo stato precedente della FSM vengono processati da una logica combinatoria(G)\\
        => Il risultato della logica viene memorizzato nella Zustandspeicher(Z)\\
        => L'uscita è esattamente la copia di tutti gli stati memorizzati(s).
    \end{minipage}%
    \hfill
    \begin{minipage}[t]{0.48\columnwidth}
        \vspace{0pt} % <-- ensures top alignment
        \includegraphics[width=\linewidth]{Images/Medwedjew.png}
    \end{minipage}

    %---------- MOORE ----------
    \subsection{Moore}
    \begin{minipage}[t]{0.48\columnwidth}
        \vspace{0pt} % <-- ensures top alignment
        Come Medwedjew ma con una logica dedicata sul ramo di output(s).\\
        Tipicamente utile per output più complessi/numerosi rispetto agli stati memorizzati(k $\neq$ n).\\
        => logica combinatoria aggiuntiva sul ramo d'uscita (F)\\
        => Più efficiente di Medwedjew per la memorizzazione degli stati
    \end{minipage}%
    \hfill
    \begin{minipage}[t]{0.48\columnwidth}
        \vspace{0pt} % <-- ensures top alignment
        \includegraphics[width=\linewidth]{Images/Moore.png}
    \end{minipage}

    %---------- MEALY ----------
    \subsection{Mealy}
    \begin{minipage}[t]{0.48\columnwidth}
        \vspace{0pt} % <-- ensures top alignment
        Si tratta della Versione Moore dove la logica sul ramo d'uscita è dipendente anche a segnali provenienti direttamente
        dagli input della FSM\\
        => Necessaria se y dipende asincronamente da delle entrate\\
        => Più complessa, spesso riducibile a Moore
    \end{minipage}%
    \hfill
    \begin{minipage}[t]{0.48\columnwidth}
        \vspace{0pt} % <-- ensures top alignment
        \includegraphics[width=\linewidth]{Images/Mealy.png}
    \end{minipage}


% ==================================== CODICE SCHELETRO FSM ====================================
    \subsection{Codice scheletro FSM (f,g,z)}


% ==================================== CODIFICA DEGLI STATI ====================================
    \subsection{Codifica degli stati}
    Gli stati di una FSM possono essere codificati in diversi modi, tra cui:
    \begin{itemize}
        \item \textbf{Codifica binaria}: ogni stato è rappresentato da un codice binario unico.
        \item \textbf{Codifica Gray}: simile alla codifica binaria, ma le transizioni tra stati adiacenti cambiano solo un bit alla volta.
        \item \textbf{Codifica one-hot}: ogni stato è rappresentato da un bit attivo, con tutti gli altri bit a zero.
        \item \textbf{Codifica one-cold}: simile alla codifica one-hot, ma solo un bit è a zero e tutti gli altri sono attivi.
    \end{itemize}