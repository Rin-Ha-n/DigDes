% ========================================= TEMPLATE INFO =========================================
%
% Author:       Rin-Ha-n
% Version:      1.0
% Last updated: 2025-05-20
% Brief:        A cheatSheet for the module DigDes (Digital Design) at OST FS25

% ================================================================================================
\documentclass[a4paper,landscape,8pt]{extarticle}
% Font size:    8pt
% Paper size:   A4
% style:        twoside (needed, so odd and even pages have different margins)
% orientation:  portrait. (use 'landscape' for landscape orientation)



% ========================================= DOCUMENT INFO =========================================
\def\title{CheatSheet Digital Design}   	     		% title
\def\shorttitle{DigDes}                              	% short title (displayed as PDF title)
\def\dozent{Prof. Dr. Paul Zbinden}                     % lecturer
\def\semester{Fs 2025}								    % semester
\def\author{Ricca Aaron}                           	    % author(s)
\def\repo{https://github.com/Rin-Ha-n/DigDes}       	% repository link
\def\version{1.0\today}                              	% version
\def\pagelimit{4}                                   	% page limit -> causes pages after limit to be red
\def\titleoption{ultra compact}                         % options: ultra compact, compact, normal
\def\enableToC{false}                                   % enable table of contents (true/false)        

\def\sectioncolor{green}                                % section color   (Used green because as shown in psycology studies it calms you) 
\def\subsectioncolor{green}                             % subsection color
\def\subsubsectioncolor{green}                          % subsubsection color

% ================================= PACKAGES, SETUP AND COMMANDS ==================================
\input{preamble.tex} % Used to include minimum required packages and settings

% ---------------------------------- Code Block word Color ----------------------------------------------
\lstdefinelanguage{VHDL}{
  morekeywords={architecture,begin,block,case,component,configuration,else,elsif,end,entity,for,function,generate,if,is,loop,package,port,process,return,signal,then,type,use,variable,wait,when,while,with,library,assert,report,after,until,others,all,constant,procedure,of,downto,to,in,out,inout,buffer,linkage,guarded,open,range,record,select,transport,reject,units,group,file,access,shared,new,next,null,exit,abs,not,and,or,xor,nand,nor,sll,srl,sla,sra,rol,ror},
  sensitive=true,
  morecomment=[l]--,
  morestring=[b]",
  morestring=[b]'
}

\lstset{
  basicstyle=\ttfamily\footnotesize,
  keywordstyle=\color{violet}\bfseries,
  commentstyle=\color{OSTDarkGreen}\itshape,
  stringstyle=\color{orange},
  numbers=left,
  numberstyle=\tiny\color{gray},
  stepnumber=1,
  numbersep=2pt,         % no space between numbers and code
  xleftmargin=1.5em,     % indent code block to keep numbers inside column
  framexleftmargin=0pt,
  breaklines=true,
  showstringspaces=false,
  tabsize=2,
  aboveskip=0pt,
  belowskip=0pt,
  resetmargins=true
}

% ---------------------------------- Code Block lines number Color ----------------------------------------------
\usepackage{xparse}
\usepackage{xstring}
\usepackage{etoolbox}
\makeatletter
\NewDocumentCommand{\highlightlines}{mmm}{%
  \StrCut{#1}{,}{\FirstLine}{\RestLines}%
  \def\@tempa{#1}%
  \edef\@tempb{\number\value{lstnumber}}%
  \def\found{0}%
  \renewcommand{\do}[1]{%
    \ifnum\pdfstrcmp{\@tempb}{##1}=0
      \def\found{1}%
    \fi
  }%
  \docsvlist{#1}%
  \ifnum\found=1
    \begingroup
      \setlength{\fboxsep}{0pt}%
      \colorbox{#2}{\strut #3}%
    \endgroup
  \else
    #3%
  \fi
}
\makeatother

% =========================================== DOCUMENT ============================================
\begin{document}
    \begin{multicols}{3}
        \raggedcolumns
        \cheatsheettitle
        \section{Introduzione}

% ==================== COMPONENT CHOICE ========================
    \subsection{Scelta/caratteristiche dei componenti}
        \noindent
        \begin{minipage}[t]{\columnwidth}
            \vspace{0pt}  %<-- ensures top alignment
            \includegraphics[width=\linewidth]{Images/DevelopmentModel.png}
        \end{minipage}
        \hfill
        \begin{minipage}[t]{\columnwidth}
            \vspace{0pt}  %<-- ensures top alignment
            \includegraphics[width=\linewidth]{Images/WhalDerRealisirungsform.png}\\[1ex]
            \includegraphics[width=\linewidth]{Images/ClassificationeDeiComponenti.png}
        \end{minipage}
        \begin{minipage}[t]{\columnwidth}
            \vspace{0pt}  % ensures top alignment
            \rotatebox{-90}{\includegraphics[height=\linewidth]{Images/DesignGuide.png}}
        \end{minipage}


% ==================== DESGIN MUSTER ========================
    \subsection{Guida al design}
        \begin{minipage}[t]{0.48\columnwidth}
            \vspace{0pt} % ensures top alignment
            \begin{enumerate}
                \item Design / Entry
                \item Funktionale Simulation
                \item Synthese
                \item Implementierung
                \begin{itemize}
                    \item Logikoptimierung
                    \item Platzierung
                    \item Verdrahtung
                \end{itemize}
                \item Timing Simulation
                \item Statische Timing Analyse
                \item Herstellungsdatenerzeugen
            \end{enumerate}
        \end{minipage}
        \hfill
        \begin{minipage}[t]{0.48\columnwidth}
            \vspace{0pt} % ensures top alignment
            \includegraphics[width=\linewidth]{Images/GuidaDesign.png}
        \end{minipage}
        \section{Programmazione VHDL}
% =========================================== LIBRARY =========================================
    \subsection{Library}
        Una libreria puo contenere componenti e o pachetti. I componenti sono descrizione di circuiti e realizzazione specifiche,
        vengono memorizzati nella libreria in modo da poter essere riutilizati piu volte e da piu progettisti contemporaneamente.\\
        I blocchi di codice di una libreria sono memorizzati in forma compilata, direttamente eseguibile.\\
        Contenuto di una libreria: Components, Packages, Functions, Procedures, Declarations.
        \begin{lstlisting}[language=VHDL]
library ieee; 
use ieee.std_logic_1164.all; -- CPP: using namespace std;
use ieee.numeric_std.all; -- Solo per operazioni aritmetiche per vettori
        \end{lstlisting}


% =========================================== ENTITY =========================================
    \subsection{entity dichiarazione}
        L'entità descrive il componente del progetto.
        In primo luogo l'entità descrive l'interfaccia (schnittstelle) del componente.
        \begin{lstlisting}[language=VHDL]
entity <entity name> is
    port (
        {<port_name> : <mode> <type>;} -- <mode> = in | out | inout
    );
end entity <entity name>;
        \end{lstlisting}

% =========================================== ARCHITECTURE ======================================
    L'architettura descrive il comportamento del componente, come funziona e come è realizzato.
    \subsection{architecture}
        \begin{lstlisting}[language=VHDL, numberstyle=\tiny\color{gray}\highlightlines{1,7,8,9}{green}]
architecture <architecture_type> of <entity_name> is
    [type_declaration]
        [component_declaration]
        [subtype_declaration]
        [constant_declaration]
        [signal_declaration]
begin
    -- codice di architettura
end <architecture_type>;
        \end{lstlisting}


% ========================================= COMPONENTI =========================================
    \subsection{component dichiarazione}
        I componenti sono utilizzati per definire le porte di un'entità, in modo da poterla utilizzare in altre entità.
        \begin{lstlisting}[language=VHDL]
component <component_name>
    port (
        {<port_name> : <mode> <type>;}
    );
end component <component_name>;
        \end{lstlisting}

        
% =========================================== PORT MAPPING ======================================
        \subsection{Port mapping}
            Il port mapping è utilizzato per collegare le porte dell'entità con i segnali dell'architettura.
            \begin{lstlisting}[language=VHDL]
U1: entity_name
    port map (
        <port_name> => <signal_name>,
        <port_name> => <signal_name>
    );
            \end{lstlisting}

% ---------------------------------------- EXAMPLE ------------------------------------
            \subsubsection{Esempio}
            \begin{lstlisting}[language=VHDL, escapeinside={(*@}{@*)}, numberstyle=\tiny\color{gray}\highlightlines{3,4,5,6,19,20,21,22,23,24}{yellow}, numberstyle=\tiny\color{gray}\highlightlines{10,11,12,13,26,27}{green}]
architecture structural of half_adder is
    -- dichiarazione del componente xor2
    component xor2
        port (
            in1, in2 : in bit;
            oup      : out bit
        );
    end component;

    -- dichiarazione del componente and2
    component and2
        port (
            in1, in2 : in bit;
            oup      : out bit
        );
    end component;

    begin
    -- instantiation of ocmponents XOR2 and AND2
    (*@\setlength{\fboxsep}{1pt}\colorbox{yellow}{\strut\textbf{\textcolor[rgb]{0.0,0.4,0.2}{-- Mappatura esplicita}}}@*)
    U1 : xor2
        port map (
            in1 => q,
            in2 => p,
            oup  => s
        );
    (*@\setlength{\fboxsep}{1pt}\colorbox{yellow}{\strut\textbf{\textcolor[rgb]{0.0,0.4,0.2}{-- Mappatura implicita}}}@*)
    U2 : and2
        port map (p, q, s) -- L'ordine delle porte segue quello della dichiarazione del componente!
            \end{lstlisting}

        
% =========================================== HIERARCHIC LEVELS ======================================
        \subsection{Hierarchie Level}
            \begin{minipage}[t]{\columnwidth}
            \vspace{0pt}  %<-- ensures top alignment
                \includegraphics[width=\linewidth]{Images/HierarchienLevel.png}
            \end{minipage}

% ============================================ TIPI =========================================
    \subsection{Tipi}
        \begin{itemize}
            \item \texttt{<architecture\_type>} = \texttt{Behavioral} \textbar{} \texttt{Structural} \textbar{} \texttt{RTL} \textbar{} \texttt{Dataflow} \textbar{} \texttt{Tb} \textbar{} \texttt{(...)} 
            \item \texttt{<mode>} = \texttt{in} \textbar{} \texttt{out} \textbar{} \texttt{inout}
            \item \texttt{<type>} = \texttt{bit} \textbar{} \texttt{bit\_vector} \textbar{} \texttt{std\_}\textcolor{red}{u}\texttt{logic} \textbar{} \texttt{std\_}\textcolor{red}{u}\texttt{logic\_vector} \textbar{} \texttt{integer} \textbar{} \texttt{boolean}
        \end{itemize}
        


% ---------------------------------------- <ARCHITECTURE TYPES> ------------------------------------
        \subsubsection{\texttt{<architecture\_type>}}
            \textbf{Behavioral}: si occupa di descrivere il comportamento del circuito, senza preoccuparsi della struttura fisica. Alto livello di astrazione concetto Wahrheitstabelle.
            \begin{lstlisting}[language=VHDL]
if rising_edge(clk) then
    if A = '1' then
    Y <= B;
    end if;
end if;
            \end{lstlisting}

            \textbf{Structural}: si occupa di descrivere la struttura fisica del circuito, utilizzando componenti e connessioni tra di essi. Medio livello di astrazione.
            \begin{lstlisting}[language=VHDL]
U1: and_gate port map (A => A, B => B, Y => Y1);
U2: or_gate  port map (A =1, B => C, Y => Y);
            \end{lstlisting}

            \textbf{RTL}: si occupa di descrivere il circuito a livello di registro e logica combinatoria, utilizzando registri e porte logiche. Basso livello di astrazione.Concetto Boolsche Ausdrucke.
            \begin{lstlisting}[language=VHDL]
if rising_edge(clk) then
    reg1 <= A and B;
    reg2 <= reg1 xor C;
end if;
            \end{lstlisting}

            \textbf{Dataflow}: si occupa di descrivere il circuito a livello di flusso di dati, utilizzando porte logiche e segnali. Basso livello di astrazione.
            \begin{lstlisting}[language=VHDL]
Y <= (A and B) or (not C);
            \end{lstlisting}

            \textbf{Tb}: si occupa di descrivere il circuito a livello di testbench, utilizzando segnali di test e componenti di test.
            \begin{lstlisting}[language=VHDL]
A <= '0'; wait for 10 ns;
A <= '1'; wait for 10 ns;
assert (Y = expected_value) report "Test failed" severity error;
            \end{lstlisting}

% ---------------------------------------- <TYPES> ------------------------------------
        \subsubsection{\texttt{<type> (dichiarazione: segnali, variabili, ...)}}
        Come vanno dichiarati tutti i segnali utilizzati internamente all'archittetura.\\Nella sintesi del codice, é vietato inizzializzare i segnali nella dichiarazione!
        \begin{itemize}
        \setlength\itemsep{0pt}
            \item \texttt{bit}: rappresenta un singolo bit, con valori \texttt{'0'} e \texttt{'1'}.
            \begin{lstlisting}[language=VHDL]
signal A : bit;
            \end{lstlisting}
            \item \texttt{bit\_vector}: rappresenta un vettore di bit, con valori \texttt{'0'} e \texttt{'1'}.
            \begin{lstlisting}[language=VHDL]
signal B : bit_vector(7 downto 0); -- vettore di 8 bit
            \end{lstlisting}
            \item \texttt{std\_logic}: rappresenta un singolo bit con valori \texttt{'0'}, \texttt{'1'}.
            \begin{lstlisting}[language=VHDL]
signal C : std_logic;
            \end{lstlisting}
            \item \texttt{std\_logic\_vector}: rappresenta un vettore di std\_logic, con valori \texttt{'0'}, \texttt{'1'}.
            \begin{lstlisting}[language=VHDL]
signal D : std_logic_vector(7 downto 0);
            \end{lstlisting}
            \item \texttt{std\_ulogic}: rappresenta un singolo bit con valori \texttt{'0'}, \texttt{'1'}, \texttt{'Z'} (alta impedenza) e \texttt{'X'} (indeterminato).
            \begin{lstlisting}[language=VHDL]
signal E : std_ulogic;
            \end{lstlisting}
            \item \texttt{std\_ulogic\_vector}: rappresenta un vettore di std\_ulogic, con valori \texttt{'0'}, \texttt{'1'}, \texttt{'Z'} e \texttt{'X'}.
            \begin{lstlisting}[language=VHDL]
signal F : std_ulogic_vector(7 downto 0);
            \end{lstlisting}
            \item \texttt{integer}: rappresenta un numero intero, con valori compresi tra $-2^{31}$ e $2^{31}-1$ (è necessario definire l'intervallo di utilizzo).
            \begin{lstlisting}[language=VHDL]
signal G : integer range 0 to 255; -- intervallo di utilizzo
            \end{lstlisting}
            \item \texttt{boolean}: rappresenta un valore booleano, con valori \texttt{true} e \texttt{false}.
            \begin{lstlisting}[language=VHDL]
signal H : boolean; -- true or false
            \end{lstlisting}
        \end{itemize}
        
% ============================================ PARALLEL SIGNAL =========================================
    \subsection{Nebenläufige Signalzuweisungen "y<=x"}
        \subsubsection{Definizione dei segnali}
            \begin{lstlisting}[language=VHDL]
singal <signal_name> :{,<singal_name>} : <type>
[:= inizianol_value]; -- inizial_value é opzionale
            \end{lstlisting}
    
    % ------------------------------ UNCONDITIONAL -----------------
        \subsubsection{Unbedingte Signalzuweisung}
            L'assegnazione dei segnali é incondizionata, quindi indipendente.
            \begin{lstlisting}[language=VHDL]
y <= '0';
y <= a and b;
            \end{lstlisting}
    
    % ------------------------------ CONDITIONAL -----------------
        \subsubsection{Bedingte Signalzuweisung}
            L'assegnazione dei segnali é eseguita in modo sequenziale, si controlla una condizione e se corretta si assegna il valore, sennó si procede con la prossima condizione.
            \begin{lstlisting}[language=VHDL]
y <= '0' when a = '1' else 
        '1' when a = '0';
            \end{lstlisting}
    
    % ------------------------------ SELECTIVE -----------------
        \subsubsection{Selektive Signalzuweisung}
            L'assegnazione dei segnali é eseguita in modo selettivo, viene selezionata il valore in base alla condizione.
            \begin{lstlisting}[language=VHDL]
with s select y <=
    '0' when "00", -- quando s = "00" y <= '0'
    '1' when "01", -- quando s = "01" y <= '1'
    'Z' when "10", -- quando s = "10" y <= 'Z'
    'X' when others; -- quando s =  altro y <= 'X'
            \end{lstlisting}
    
    % ------------------------------ AGGREGATE -----------------
        \subsubsection{aggregate}
            L'aggregazione dei segnali permette di aggregare segnali individuali in un unico segnale.
            \begin{lstlisting}[language=VHDL]
y <= (a, b, '1', '0'); -- Assegnazione implicita
y <= (0 => '0', 1 => '1', 2 => b, others => a); -- Assegnazione esplicita (<posizione_vettoriale> => <valore>)
            \end{lstlisting}
    
    % ------------------------------ CONCATENATE -----------------
        \subsubsection{concatenate}
            La concatenazione dei segnali permette di concatenare segnali in un unico segnale.
            \begin{lstlisting}[language=VHDL]
y <= v_1 & v_2;
            \end{lstlisting}


% ========================================= PROCESSI =========================================
    \subsection{Nebenläufige Prozesse}
        I processi sono "Nebenläufige" di conseguenza iniziano ad essere eseguiti in concorrenza. Ma all'interno il codice viene eseguito normalmente (istruzioni sequenziali, sequenzialmente. istruzioni parallele in modo parallelo).\\
        I processi sono sezioni di codice che vengono eseguite ogni volta che un \textcolor{red}{Segnale sensibile} nella lista sensibile (Sensitivitätliste) cambia di stato.         % Inizia un codice blocco, con colori VHDL (vedi definizione in DigDes.tex), escape inside is used to use external commandstf
        \begin{lstlisting}[language=VHDL, escapeinside={(*@}{@*)}, numberstyle=\tiny\color{gray}\highlightlines{3,4,5,6,7}{green}]
process ((*@\textcolor{red}{clk, reset}@*))
    begin
        if reset = '1' then
            -- inserisci il codice da eseguire in caso di reset
        elsif rising_edge(clk) then
            -- inserisci il codice da eseguire ad ogni fronte di salita del clock
        end if;
    end process;
        \end{lstlisting}
    
    % ------------------------------ SEQUENZIELLE ANWEISUNGEN IM PROZESSE -----------------
        \subsubsection{sequenzielle Anweisungen im Prozesse}
            Le istruzioni che vengono eseguite strettamente sequenzialmente all'interno di un processo sono:
            \begin{lstlisting}[language=VHDL]
-- struttura if else:
if condition_a then
    {sequential statements}
elsif condition_b then
    {sequential statements}
else
    {sequential statements}
end if;


-- struttura case when:
case expression is
    when choice_a => {sequential statements}
    when choice_b => {sequential statements}
    when others => {sequential statements}
end case;
            \end{lstlisting}
    
    % ------------------------------ Eigenschaften nebenläufiger Prozesse -----------------
        \subsubsection{Eigenschaften nebenläufiger Prozesse}
        Le proprietà più importanti, ovvero le estensioni rispetto alle assegnazioni di segnale, possono essere così riassunte:
            \begin{itemize}
                \item I processi possono assegnare due o più segnali contemporaneamente.
                \item L'elaborazione delle informazioni per l'assegnazione dei segnali avviene in una sequenza di comandi che vengono eseguiti uno dopo l'altro (procedurale).
                \item I processi permettono l'uso di variabili per la memorizzazione temporanea dei valori dei segnali.
                \item Grazie all'uso delle liste di sensibilità è garantito un miglior controllo sulle condizioni di esecuzione della parte di codice.
            \end{itemize}
    
    % ------------------------------ Variablen in nebenläufigen Prozessen -----------------
        \subsubsection{Variablen in nebenläufigen Prozessen}
        Le variabili offrono due opzioni utili nei processi:
            \begin{itemize}
                \item Accesso a un valore aggiornato all'interno del processo stesso.
                \item Preparazione di un'espressione di controllo, ad esempio per un "case when".
            \end{itemize}
            Le variabili sono dichiarate all'interno dei processi e sono visibili esclusivamente all'interno degli stessi.
            Il valore assegnato può essere letto immediatamente.\\
            L'assegnazione di valore a una variabile avviene con l'operatore :=, a differenza dell'assegnazione ai segnali che utilizza <=.
        \begin{lstlisting}[language=VHDL]
variable <var_name> {,var_name}: <type> [:= expression];
        \end{lstlisting}
        \section{State machine}
    Le \textbf{Finite State Machine} (FSM) sono circuiti sequenziali che possono essere in uno stato tra un insieme finito di stati. 
    La transizione tra gli stati avviene in base a segnali di ingresso e può essere condizionata da segnali di clock e reset.

    \begin{minipage}[t]{1\columnwidth}
        \vspace{0pt} % <-- ensures top alignment
        \includegraphics[width=\linewidth]{Images/FSMGeneralizzata.png}
    \end{minipage}%
    

% ==================================== CODICE SCHELETRO FSM ====================================
    \subsection{Codice scheletro FSM (f,g,z)}


% ==================================== CODIFICA DEGLI STATI ====================================
    \subsection{Codifica degli stati}
    Gli stati di una FSM possono essere codificati in diversi modi, tra cui:
    \begin{itemize}
        \item \textbf{Codifica binaria}: ogni stato è rappresentato da un codice binario unico.
        \item \textbf{Codifica Gray}: simile alla codifica binaria, ma le transizioni tra stati adiacenti cambiano solo un bit alla volta.
        \item \textbf{Codifica one-hot}: ogni stato è rappresentato da un bit attivo, con tutti gli altri bit a zero.
        \item \textbf{Codifica one-cold}: simile alla codifica one-hot, ma solo un bit è a zero e tutti gli altri sono attivi.
    \end{itemize}
    \end{multicols}
\end{document}

% TODO:
% rimuovere tabs tra linee numerate e il codice
% aggiungiere comando per le linee colorate in modo che posso colorarne diverse con colori diversi (cheatsheettemplate)
% aggiungere al comando titolo animegirl ganbatteneeee!! (cheatsheettemplate)