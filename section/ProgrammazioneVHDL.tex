\section{Programmazione VHDL}
% =========================================== LIBRARY =========================================
    \subsection{Library}
        Una libreria puo contenere componenti e o pachetti. I componenti sono descrizione di circuiti e realizzazione specifiche,
        vengono memorizzati nella libreria in modo da poter essere riutilizati piu volte e da piu progettisti contemporaneamente.\\
        I blocchi di codice di una libreria sono memorizzati in forma compilata, direttamente eseguibile.\\
        Contenuto di una libreria: Components, Packages, Functions, Procedures, Declarations.
        \begin{lstlisting}[language=VHDL]
library ieee; 
use ieee.std_logic_1164.all; -- CPP: using namespace std;
use ieee.numeric_std.all; -- Solo per operazioni aritmetiche per vettori
        \end{lstlisting}


% =========================================== ENTITY =========================================
    L'entità descrive il componente del progetto.
    In primo luogo l'entità descrive l'interfaccia (schnittstelle) del componente.
    \subsection{entity}
        \begin{lstlisting}[language=VHDL]
entity <entity name> is
    port (
        {<port_name> : <mode> <type>;}
    );
end entity <entity name>;
        \end{lstlisting}

% =========================================== ARCHITECTURE ======================================
    L'architettura descrive il comportamento del componente, come funziona e come è realizzato.
    \subsection{architecture}
        \begin{lstlisting}[language=VHDL, numberstyle=\tiny\color{gray}\highlightlines{1,7,8,9}{green}]
architecture <architecture_type> of <entity_name> is
    [type_declaration]
        [subtype_declaration]
        [constant_declaration]
        [signal_declaration]
        [component_declaration]
begin
    -- codice di architettura
end <architecture_type>;
        \end{lstlisting}

        
% =========================================== PORT MAPPING ======================================
        \subsection{Port mapping}
            Il port mapping è utilizzato per collegare le porte dell'entità con i segnali dell'architettura.
            \begin{lstlisting}[language=VHDL]
U1: entity_name
    port map (
        <port_name> => <signal_name>,
        <port_name> => <signal_name>
    );
            \end{lstlisting}

% ---------------------------------------- EXAMPLE ------------------------------------
            \subsubsection{Esempio}
            \begin{lstlisting}[language=VHDL, numberstyle=\tiny\color{gray}\highlightlines{3,4,5,6,19,20,21,22,23,24}{yellow}, numberstyle=\tiny\color{gray}\highlightlines{10,11,12,13,26,27}{green}]
architecture structural of half_adder is
    component xor2
        port (
            in1, in2 : in bit;
            oup      : out bit
        );
    end component;

    component and2
        port (
            in1, in2 : in bit;
            oup      : out bit
        );
    end component;

    begin
    -- instantiation of ocmponents XOR2 and AND2
    -- Mappatura esplicita
    U1 : xor2
        port map (
            in1 => q,
            in2 => p,
            oup  => s
        );
    -- Mappatura implicita
    U2 : and2
        port map (p, q, s) -- L'ordine delle porte segue quello della dichiarazione del componente!
            \end{lstlisting}


% ========================================= COMPONENTI =========================================
    \subsection{component}
        I componenti sono utilizzati per definire le porte di un'entità, in modo da poterla utilizzare in altre entità.
        \begin{lstlisting}[language=VHDL]
component <component_name> is
    port (
        {<port_name> : <mode> <type>;}
    );
end component <component_name>;
        \end{lstlisting}


% ========================================= PROCESSI =========================================
    \subsection{Processi}
        I processi sono sezioni di codice che vengono eseguite ogni volta che un \textcolor{red}{Segnale sensibile} nella lista sensibile cambia di stato.
        % Inizia un codice blocco, con colori VHDL (vedi definizione in DigDes.tex), escape inside is used to use external commandstf
        \begin{lstlisting}[language=VHDL, escapeinside={(*@}{@*)}, numberstyle=\tiny\color{gray}\highlightlines{3,4,5,6,7}{green}]
process ((*@\textcolor{red}{clk, reset}@*))
    begin
        if reset = '1' then
            -- codice di reset
        elsif rising_edge(clk) then
            -- codice di clock
        end if;
    end process;
        \end{lstlisting}


% ============================================ TIPI =========================================
    \subsection{Tipi}
        \begin{itemize}
            \item \texttt{<architecture\_type>} = \texttt{Behavioral} \textbar{} \texttt{Structural} \textbar{} \texttt{RTL} \textbar{} \texttt{Dataflow} \textbar{} \texttt{Tb}
            \item \texttt{<mode>} = \texttt{in} \textbar{} \texttt{out} \textbar{} \texttt{inout}
            \item \texttt{<type>} = \texttt{bit} \textbar{} \texttt{bit\_vector} \textbar{} \texttt{std\_}\textcolor{red}{u}\texttt{logic} \textbar{} \texttt{std\_}\textcolor{red}{u}\texttt{logic\_vector} \textbar{} \texttt{integer} \textbar{} \texttt{boolean}
        \end{itemize}
        


% ---------------------------------------- <ARCHITECTURE TYPES> ------------------------------------
        \subsubsection{\texttt{<architecture\_type>}}
            \textbf{Behavioral}: si occupa di descrivere il comportamento del circuito, senza preoccuparsi della struttura fisica. Alto livello di astrazione.
            \begin{lstlisting}[language=VHDL]
if rising_edge(clk) then
    if A = '1' then
    Y <= B;
    end if;
end if;
            \end{lstlisting}

            \textbf{Structural}: si occupa di descrivere la struttura fisica del circuito, utilizzando componenti e connessioni tra di essi. Medio livello di astrazione.
            \begin{lstlisting}[language=VHDL]
U1: and_gate port map (A => A, B => B, Y => Y1);
U2: or_gate  port map (A =1, B => C, Y => Y);
            \end{lstlisting}

            \textbf{RTL}: si occupa di descrivere il circuito a livello di registro e logica combinatoria, utilizzando registri e porte logiche. Basso livello di astrazione.
            \begin{lstlisting}[language=VHDL]
if rising_edge(clk) then
    reg1 <= A and B;
    reg2 <= reg1 xor C;
end if;
            \end{lstlisting}

            \textbf{Dataflow}: si occupa di descrivere il circuito a livello di flusso di dati, utilizzando porte logiche e segnali. Basso livello di astrazione.
            \begin{lstlisting}[language=VHDL]
Y <= (A and B) or (not C);
            \end{lstlisting}

            \textbf{Tb}: si occupa di descrivere il circuito a livello di testbench, utilizzando segnali di test e componenti di test.
            \begin{lstlisting}[language=VHDL]
A <= '0'; wait for 10 ns;
A <= '1'; wait for 10 ns;
assert (Y = expected_value) report "Test failed" severity error;
            \end{lstlisting}

% ---------------------------------------- <TYPES> ------------------------------------
        \subsubsection{\texttt{<type>}}
        \begin{itemize}
        \setlength\itemsep{0pt}
            \item \texttt{bit}: rappresenta un singolo bit, con valori \texttt{'0'} e \texttt{'1'}.
            \begin{lstlisting}[language=VHDL]
signal A : bit;
            \end{lstlisting}
            \item \texttt{bit\_vector}: rappresenta un vettore di bit, con valori \texttt{'0'} e \texttt{'1'}.
            \begin{lstlisting}[language=VHDL]
signal B : bit_vector(7 downto 0); -- vettore di 8 bit
            \end{lstlisting}
            \item \texttt{std\_logic}: rappresenta un singolo bit con valori \texttt{'0'}, \texttt{'1'}.
            \begin{lstlisting}[language=VHDL]
signal C : std_logic;
            \end{lstlisting}
            \item \texttt{std\_logic\_vector}: rappresenta un vettore di std\_logic, con valori \texttt{'0'}, \texttt{'1'}.
            \begin{lstlisting}[language=VHDL]
signal D : std_logic_vector(7 downto 0);
            \end{lstlisting}
            \item \texttt{std\_ulogic}: rappresenta un singolo bit con valori \texttt{'0'}, \texttt{'1'}, \texttt{'Z'} (alta impedenza) e \texttt{'X'} (indeterminato).
            \begin{lstlisting}[language=VHDL]
signal E : std_ulogic;
            \end{lstlisting}
            \item \texttt{std\_ulogic\_vector}: rappresenta un vettore di std\_ulogic, con valori \texttt{'0'}, \texttt{'1'}, \texttt{'Z'} e \texttt{'X'}.
            \begin{lstlisting}[language=VHDL]
signal F : std_ulogic_vector(7 downto 0);
            \end{lstlisting}
            \item \texttt{integer}: rappresenta un numero intero, con valori compresi tra $-2^{31}$ e $2^{31}-1$ (è necessario definire l'intervallo di utilizzo).
            \begin{lstlisting}[language=VHDL]
signal G : integer range 0 to 255; -- intervallo di utilizzo
            \end{lstlisting}
            \item \texttt{boolean}: rappresenta un valore booleano, con valori \texttt{true} e \texttt{false}.
            \begin{lstlisting}[language=VHDL]
signal H : boolean; -- true or false
            \end{lstlisting}
        \end{itemize}