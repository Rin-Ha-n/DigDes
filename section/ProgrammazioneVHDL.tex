\section{Programmazione VHDL}
% =========================================== ENTITY =========================================
    \subsection{entity}
        \begin{lstlisting}[language=VHDL, escapeinside={(*@}{@*)}]
            entity <entity name> is
                port (
                    {<port_name> : <mode> <type>;}
                );
            end entity <entity name>;

            mode = in | out | inout | buffer
        \end{lstlisting}

% =========================================== ARCHITECTURE ======================================
    \subsection{architecture}
        \begin{lstlisting}[language=VHDL, escapeinside={(*@}{@*)}, numberstyle=\tiny\color{gray}\highlightlines{3}{6}{yellow}]
            architecture <architecture_type> of <entity_name> is
                [type_declaration]
                    [subtype_declaration]
                    [constant_declaration]
                    [signal_declaration]
                    [component_declaration]
            begin
                -- codice di architettura
            end <architecture_type>;

            architecture_type = Behavioral | Structural | RTL | Dataflow | Tb
        \end{lstlisting}

% ----------------------------------------ARCHITECTURE TYPES ------------------------------------
        \subsubsection{Tipi di architettura}
            Behavioral: si occupa di descrivere il comportamento del circuito, senza preoccuparsi della struttura fisica.
            \begin{lstlisting}[language=VHDL]
                if rising_edge(clk) then
                    if A = '1' then
                    Y <= B;
                    end if;
                end if;
                \end{lstlisting}


            Structural: si occupa di descrivere la struttura fisica del circuito, utilizzando componenti e connessioni tra di essi.
            \begin{lstlisting}[language=VHDL]
                U1: and_gate port map (A => A, B => B, Y => Y1);
                U2: or_gate  port map (A =1, B => C, Y => Y);
                \end{lstlisting}


            RTL: si occupa di descrivere il circuito a livello di registro e logica combinatoria, utilizzando registri e porte logiche.
            \begin{lstlisting}[language=VHDL]
                if rising_edge(clk) then
                    reg1 <= A and B;
                    reg2 <= reg1 xor C;
                end if;
                \end{lstlisting}


            Dataflow: si occupa di descrivere il circuito a livello di flusso di dati, utilizzando porte logiche e segnali.
            \begin{lstlisting}[language=VHDL]
                Y <= (A and B) or (not C);
                \end{lstlisting}


            Tb: si occupa di descrivere il circuito a livello di testbench, utilizzando segnali di test e componenti di test.
            \begin{lstlisting}[language=VHDL]
                A <= '0'; wait for 10 ns;
                A <= '1'; wait for 10 ns;
                assert (Y = expected_value) report "Test failed" severity error;
                \end{lstlisting}

% ==================================== CODICE SCHELETRO FSM ====================================
    \subsection{Codice scheletro FSM (f,g,z)}

% ========================================= PROCESSI =========================================
    \subsection{Processi}
        I processi sono sezioni di codice che vengono eseguite ogni volta che un \textcolor{red}{Segnale sensibile} nella lista sensibile cambia di stato.
        % Inizia un codice blocco, con colori VHDL (vedi definizione in DigDes.tex), escape inside is used to use external commandstf
        \begin{lstlisting}[language=VHDL, escapeinside={(*@}{@*)}, numberstyle=\tiny\color{gray}\highlightlines{3}{7}{yellow}]
        process ((*@\textcolor{red}{clk, reset}@*))
            begin
                if reset = '1' then
                    -- codice di reset
                elsif rising_edge(clk) then
                    -- codice di clock
                end if;
            end process;
        \end{lstlisting}