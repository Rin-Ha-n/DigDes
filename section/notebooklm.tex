\section{notebooklm.tex}

\textbf{\texttt{std\_ulogic} (unresolved)} [8]:
\begin{itemize}
    \item Replaces \texttt{bit} type [8].
    \item \textbf{Allows only one driver per signal}; compiler reports error on multiple drivers [8].
    \item Suitable for three-state outputs using 'Z' [9].
    \item Recognises uninitialized ('U'), weak ('H', 'L'), and don't care ('-') states [9, 10].
    \item Recommended for educational projects for its strictness [11].
\end{itemize}

\textbf{\texttt{std\_logic} (resolved)} [10]:
\begin{itemize}
    \item Removes the 'unresolved' restriction, allowing multiple drivers [10].
    \item \textbf{Simulator automatically resolves driver conflicts} based on a resolution function (e.g., '0' and '1' resolving to 'X', '0' and 'Z' to '0') [10, 12].
    \item More prone to hidden errors (compiler won't catch multiple assignments) [13].
    \item Widely used in practice for its flexibility, especially for physical implementation concerns [14].
    \item In VHDL 2008, \texttt{std\_logic} is a subtype of \texttt{std\_ulogic} [15].
\end{itemize}
\subsubsection{Modelling Bidirectional Buses in VHDL}
\begin{itemize}[leftmargin=1.5em]
    \item Uses the \textbf{\texttt{inout}} port mode.
    \item Writing to the bus: \verb|bus_i_o <= out_i when enable = '1' else (others => 'Z');|
    \item Reading from the bus: \verb|in_i <= bus_i_o;|
\end{itemize}

\subsection{VHDL Key Concept IV: Event-Based Concept for Temporal Representation in Simulations}

\subsubsection{Simulation as a Central Step}
\begin{itemize}[leftmargin=1.5em]
    \item Crucial for design verification.
    \item \textbf{Reusable Test-Benches} help with regression testing.
    \item \textbf{Later error detection is more costly.}
\end{itemize}

\subsubsection{Modelling Temporal Behaviour}
\begin{itemize}[leftmargin=1.5em]
    \item \textbf{Event Queue:} Simulators use an event queue to process events in order.
    \item \textbf{Delta-Time Model:}
    \begin{itemize}
        \item Functional simulation model.
        \item Infinitesimal time delay (\textit{delta cycle}).
        \item Ensures cause-effect separation in waveform views.
    \end{itemize}
    \item \textbf{Transport Delay:}
    \begin{itemize}
        \item Models real delay.
        \item Syntax: \verb|B <= transport A after tp;|
        \item \textbf{All pulses propagate}.
    \end{itemize}
    \item \textbf{Inertial Delay (default):}
    \begin{itemize}
        \item Models physical filtering of glitches.
        \item Syntax: \verb|B <= A after tp;|
        \item \textbf{Short pulses filtered out}.
    \end{itemize}
\end{itemize}

\subsubsection{Simulation Setup (Test-Bench)}
\begin{itemize}[leftmargin=1.5em]
    \item Not synthesizable.
    \item \textbf{Device Under Test (DUT)}: Component under test.
    \item \textbf{Stimulus Generator}: Drives input signals.
    \item \textbf{Response Monitor}: Verifies DUT output.
    \item \textbf{\texttt{ASSERT}} Statement:
    \begin{itemize}
        \item Syntax: \verb|assert (cond) report "msg" severity level;|
        \item Levels: \texttt{note}, \texttt{warning}, \texttt{error}, \texttt{failure}.
    \end{itemize}
\end{itemize}

\subsection{VHDL Key Concept VIII: Arithmetic and Data Types}

\subsubsection{Arithmetic Operations}
\begin{itemize}[leftmargin=1.5em]
    \item Problem: \texttt{bit\_vector}, \texttt{std\_logic\_vector} don't support arithmetic.
    \item Solution: Use \texttt{ieee.numeric\_std}.
    \item Defines:
    \begin{itemize}
        \item \textbf{\texttt{signed}}: For signed numbers.
        \item \textbf{\texttt{unsigned}}: For unsigned numbers.
    \end{itemize}
    \item Functions: \texttt{+}, \texttt{-}, \texttt{}, \texttt{abs}, \texttt{/}, \texttt{mod}, \texttt{rem}, \texttt{**} (limited).
    \item Prefer \texttt{numeric\_std} over older \texttt{std\_logic\_signed} or \texttt{unsigned}.
\end{itemize}

\subsubsection{Data Types}
\begin{itemize}[leftmargin=1.5em]
    \item Strong typing enforces correctness.
    \item \textbf{Scalar Types:}
    \begin{itemize}
        \item Discrete: \texttt{integer}, \texttt{boolean}, \texttt{bit}, \texttt{std\_ulogic}, etc.
        \item Physical: \texttt{time}, etc.
    \end{itemize}
    \item \textbf{Composite Types:}
    \begin{itemize}
        \item Arrays: \texttt{std\_logic\_vector}, etc.
        \item Records: Heterogeneous collections.
    \end{itemize}
\end{itemize}

\subsubsection{Type Conversion}
\begin{itemize}[leftmargin=1.5em]
    \item \textbf{\texttt{resize}}: For \texttt{signed}/\texttt{unsigned} vectors.
    \item \textbf{Type Casting:}
    \begin{itemize}
        \item Syntax: \verb|target_type(signal)|
        \item Example: \verb|o_int <= unsigned(a) + unsigned(b);|
    \end{itemize}
    \item \textbf{Type Conversion Functions:}
    \begin{itemize}
        \item Between different base types.
        \item Examples: \verb|to_unsigned(int, len)|, \verb|to_integer(vec)|
    \end{itemize}
\end{itemize}

\subsection{VHDL Key Concept IX: Parametrisability of Models}

\subsubsection{Reusability}
\begin{itemize}[leftmargin=1.5em]
    \item \textbf{Essential for efficient design}.
    \item General models can be adapted via parameters.
\end{itemize}

\subsubsection{Using \texttt{generic}}
\begin{itemize}[leftmargin=1.5em]
    \item Allows time-invariant parameters in \texttt{entity}.
    \item Example: \verb|generic(max_count: integer := 127);|
    \item Used for widths, delays, strengths.
    \item Assigned via \texttt{generic map}:
    \begin{itemize}
        \item \verb|instance_name : component_name generic map (...) port map (...);|
    \end{itemize}
\end{itemize}

\subsubsection{Timing of Parameter Fixation}
\begin{itemize}[leftmargin=1.5em]
    \item \textbf{Design-time}: via \texttt{constant}.
    \item \textbf{Compile-time}: via \texttt{generic}.
    \item \textbf{Runtime}: via \texttt{signal} assignments.
\end{itemize}
