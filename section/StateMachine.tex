\section{State machine}
    \begin{minipage}[t]{0.48\columnwidth}
        \vspace{0pt} % <-- ensures top alignment
        Le \textbf{Finite State Machine} (FSM) sono macchine a base di circuiti logici sequenzali.\\
        Sono in grado quindi 
        di eseguire operazioni logiche e di poterle memorizzare in modo da consegnare in uscita una funzione
        che è dipendente dallo stato attuale (memorizzato con gli input precedenti) e opzionalmente anche dagli input attuali(Mealy).\\
        Le tre alternative proposte di seguito sono delle possibilità di incapsulamento standardizzato
        della funzione desiderata, qualunque essa sia.\\
        - Cit. Alessio Ciceri
    \end{minipage}
    \hfill
    \begin{minipage}[t]{0.48\columnwidth}
        \vspace{0pt} % <-- ensures top alignment
        \includegraphics[width=\linewidth]{Images/FSMGeneralizzata.png}
    \end{minipage}%


% ==================== BUBBLE DIAGRAM ========================
    \subsection{Bubble diagram}
        \begin{minipage} [t]{0.48\columnwidth}
            \vspace{0pt} % ensures top alignment
            \begin{itemize}
                \item \textbf{Bolle}: Ogni bolla rappresenta uno stato
                \item \textbf{Freccie}: Condizione per passare da uno stato all'altro dev'essere scritta accanto alla freccia.
                \item \textbf{Moore}: Gli output sono associati agli stati, quindi scritti dentro a quest'ultimi.
                \item \textbf{Melay}: Gli output sono associati alle transizioni, quindi scritti accanto alle frecce.
            \end{itemize}
        \end{minipage}
        \begin{minipage} [t]{0.48\columnwidth}
            \vspace{0pt} % ensures top alignment
            \includegraphics[width=\linewidth]{Images/BubbleDiagram.png}
        \end{minipage}
    

% ==================================== FAMIGLIE FSM ====================================
    %---------- MEDWEJDJEW ----------
    \subsection{Medwedjew (sincrona)}
    \begin{minipage}[t]{0.48\columnwidth}
        \vspace{0pt} % <-- ensures top alignment
        Composta da un blocco di logica combinatoria(G) che risolve la funzione desiderata e da un blocco di memoria(Z)
        che memorizza gli stati.\\ 
        => Gli Input e lo stato attuale della FSM vengono processati da una logica combinatoria(G)\\
        => Il risultato della logica viene memorizzato nella Zustandspeicher(Z)\\
        => L'uscita è esattamente la copia di tutti gli stati memorizzati(s).
    \end{minipage}%
    \hfill
    \begin{minipage}[t]{0.48\columnwidth}
        \vspace{0pt} % <-- ensures top alignment
        \includegraphics[width=\linewidth]{Images/Medwedjew.png}
    \end{minipage}

    %---------- MOORE ----------
    \subsection{Moore (sincrona)}
    \begin{minipage}[t]{0.48\columnwidth}
        \vspace{0pt} % <-- ensures top alignment
        Come Medwedjew ma con una logica dedicata sul ramo di output(s).\\
        Tipicamente utile per output più complessi/numerosi rispetto agli stati memorizzati(k $\neq$ n).\\
        => logica combinatoria aggiuntiva sul ramo d'uscita (F)\\
        => Più efficiente di Medwedjew per la memorizzazione degli stati
    \end{minipage}%
    \hfill
    \begin{minipage}[t]{0.48\columnwidth}
        \vspace{0pt} % <-- ensures top alignment
        \includegraphics[width=\linewidth]{Images/Moore.png}
    \end{minipage}

    %---------- MEALY ----------
    \subsection{Mealy (sincrona e asincrona in F)}
    \begin{minipage}[t]{0.48\columnwidth}
        \vspace{0pt} % <-- ensures top alignment
        Si tratta della Versione Moore dove la logica sul ramo d'uscita è dipendente anche a segnali provenienti direttamente
        dagli input della FSM\\
        => Necessaria se y dipende asincronamente da delle entrate\\
        => Più complessa.
    \end{minipage}%
    \hfill
    \begin{minipage}[t]{0.48\columnwidth}
        \vspace{0pt} % <-- ensures top alignment
        \includegraphics[width=\linewidth]{Images/Mealy.png}
    \end{minipage}


% ==================================== CODICE SCHELETRO FSM ====================================
    \subsection{Codice scheletro FSM (G,Z,F)}
        \begin{lstlisting}[language=VHDL, escapeinside={(*@}{@*)}]

G: process(present_state, inputs)
begin
    next_state <= de fault_state;
        case present_state is
            when X_state =>
                next_state <= Y_state;
            when others =>
                next_state <= R_state;
    end case;
end process;

Z: process(clk)
begin
    if clk(*@'@*)event and clk = '1' then
        if reset = '1' then
            present_state <= reset_state;
        else
            present_state <= next_state;
        end if;
    end if;
end process;

F: process(present_state, …)
begin
    oup <= default_value;
    case present_state is
        when X_state =>
            oup <= "1001";
        when others =>
            oup <= "1111";
    end case;
end process;

        \end{lstlisting}


% ==================================== CODIFICA DEGLI STATI ====================================
    \subsection{Codifica degli stati (Z-Register)}
    La dimenzione del registro $ = 2^{n->bit}$
    Gli stati di una FSM possono essere codificati in diversi modi, tra cui:
    \begin{itemize}
        \item \textbf{Codifica binaria}: ogni stato è rappresentato da un codice binario unico.
        \item \textbf{Codifica Gray}: simile alla codifica binaria, ma le transizioni tra stati adiacenti cambiano solo un bit alla volta.
        \item \textbf{Codifica one-hot}: ogni stato è rappresentato da un bit attivo, con tutti gli altri bit a zero.
        \item \textbf{Codifica one-cold}: simile alla codifica one-hot, ma solo un bit è a zero e tutti gli altri sono attivi.
    \end{itemize}